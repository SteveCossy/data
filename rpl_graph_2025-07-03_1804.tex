% Generated by RPL Log Parser
\documentclass[tikz]{standalone}
\usetikzlibrary{graphs,graphdrawing}
\usegdlibrary{layered}
\begin{document}

\begin{tikzpicture}[rpl-node/.style={circle,draw=blue!50,fill=blue!20,thick,minimum size=7mm},rpl-root/.style={rpl-node,fill=red!30,double,draw=red!50},every edge/.style={draw=black!60,->,>=stealth}]
\graph [layered layout, nodes={rpl-node}] {
  1;
  11;
  13;
  14;
  15;
  16;
  17;
  18;
  19;
  2;
  20;
  21;
  22;
  23;
  3;
  4 [rpl-root];
  5;
  4 -> 23;
  5 -> 23;
  13 -> 23;
  14 -> 23;
  18 -> 23;
  21 -> 5;
  23 -> 18;
  15 -> 23;
  3 -> 4;
  11 -> 22;
  17 -> 4;
  19 -> 22;
  20 -> 22;
  16 -> 17;
  22 -> 4;
  2 -> 20;
  1 -> 2;
};
\end{tikzpicture}
\begin{tikzpicture}[rpl-node/.style={circle,draw=blue!50,fill=blue!20,thick,minimum size=7mm},rpl-root/.style={rpl-node,fill=red!30,double,draw=red!50},every edge/.style={draw=black!60,->,>=stealth}]
\graph [layered layout, nodes={rpl-node}] {
  1;
  13;
  14;
  16;
  17;
  18;
  19;
  21;
  22;
  23;
  3;
  4;
  5;
  6 [rpl-root];
  9 [rpl-root];
  4 -> 9;
  5 -> 9;
  13 -> 9;
  14 -> 9;
  18 -> 9;
  21 -> 9;
  23 -> 9;
  3 -> 4;
  17 -> 4;
  19 -> 4;
  22 -> 4;
  16 -> 19;
  1 -> 6;
};
\end{tikzpicture}

\end{document}
